% \chapter*{Daftar Istilah}
\clearpage
\begin{center}
    \smallskip
    \large \bfseries{Daftar Istilah}

    \begin{table}[h]
        \begin{tabularx}{\textwidth}{|l|X|}
        \textbf{Dataset}     & Kumpulan data yang digunakan untuk melakukan pelatihan, validasi, maupun evaluasi      \\
        \textbf{Model}       & Representasi matematika yang didapat dari hasil pembelajaran menggunakan data latih               \\
        \textbf{Baseline}    & Performa model atau model yang dijadikan acuan dasar                                              \\
        \textbf{Arsitektur}  & Struktur model yang terdiri dari berbagai macam rule, fungsionalitas, dan implementasi            \\
        \textbf{Transformer} & Arsitektur yang pertama kali dideskripsikan oleh (Vaswani et al., 2017) untuk memodelkan sekuens. \\
        \textbf{LearningRate} & Besar perubahan yang dilakukan ke model pada setiap iterasi pembelajaran \\ 
        \textbf{Callback} & Fungsi yang melekat pada fase pembelajaran model \\
        \textbf{EarlyStopping} & Callback yang akan memberhentikan pembelajaran ketika kondisi yang ditentukan telah dipenuhi \\
        \textbf{ReduceeLrOnPlateau} & Callback yang akan menurunkan besar LearningRate ketika model sudah tidak belajar lagi berdasarkan kondisi ditentukan. \\
        \textbf{Fine-tune} & Proses melatih kembali model ke permasalahan spesifik dari model yang sebelumnya sudah dilatih pada data umum \\
        \end{tabularx}
    \end{table}
\end{center}
\clearpage
