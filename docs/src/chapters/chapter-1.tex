\chapter{Pendahuluan}

Bab pendahuluan ini menjelaskan tentang landasan pembuatan tugas akhir mengenai klasifikasi teks berbahasa Indonesia menggunakan \textit{multi-lingual language model}. Bab ini terdiri dari latar belakang, rumusan masalah, tujuan, batasan masalah, metodologi, dan jadwal pelaksanaan tugas akhir.

\section{Latar Belakang}

Permasalahan klasifikasi pembelajaran mesin \textit{supervised} dapat digambarkan sebagai berikut: diberikan satu set label klasifikasi C, dan satu set contoh pelatihan E, yang masing-masing telah diberi salah satu label kelas dari C, sistem harus menggunakan contoh pelatihan E untuk membentuk hipotesis yang dapat digunakan untuk memprediksi label kelas dari contoh baru yang tak digunakan untuk melatih \parencite{mitchell_machine_1997}. Dalam permasalahan klasifikasi teks, data pelatihan E merupakan teks seperti dokumen atau komentar daring pada media sosial. Dalam tugas akhir ini, permasalahan yang akan menjadi fokus adalah analisis sentimen dan klasifikasi ujaran kebencian \& kasar.

Analisis sentimen adalah proses pendeteksian dan pengekstraksian informasi subjektif mengenai sentimen dalam sebuah teks. Hal ini dapat dilakukan pada beberapa level ekstraksi yaitu pada level dokumen, kalimat, hingga level spesifik terkait aspek tertentu \parencite{Liu2012}. Pada level paling granular, analisis sentimen dilakukan pada level aspek. Pada level kalimat, sentimen ditentukan untuk setiap kalimat meskipun dalam satu kalimat dapat memiliki lebih dari satu aspek. Dan terakhir pada level dokumen, analisis sentimen dilakukan secara keseluruhan meskipun dalam satu dokumen dapat memiliki lebih dari satu kalimat dan aspek sentimen. Meski analisis sentimen pada level lebih granular dapat memberikan analisa lebih detail, analisis sentimen pada level dokumen masih banyak digunakan untuk mengetahui sentimen secara keseluruhan.

Klasifikasi ujaran kebencian \& kasar adalah proses mengategorikan sebuah teks, biasanya komentar di sosial media atau web, berdasarkan masuk atau tidaknya teks tersebut dalam definisi ujaran yang mengandung kebencian \& kasar. Hal ini dapat dilakukan secara biner dan multi-kelas / multi-label. Pada kasus biner, teks hanya dikategorikan sebagai ujaran yang mengandung kebencian \& kasar atau tidak mengandung sama sekali. Sedangkan pada kasus multi-kelas / multi-label, teks selanjutnya dianalisa untuk mengetahui siapa targetnya atau seberapa parah ujaran kebencian \& kasarnya.

Analisis sentimen dan klasifikasi ujaran kebencian \& kasar dapat dilakukan dengan pendekatan berbasis aturan atau statistik. Dalam sentimen analisisis, pendekatan berbasis aturan seperti VADER \parencite{VADER} memanfaatkan kamus kata sentimen untuk menilai sentimen suatu dokumen. Begitu juga dalam klasifikasi ujaran kebencian \& kasar, \parencite{lexicon_hatespeech_2015} memanfaatkan kamus yang berisi kata-kata negatif dan kebencian. Pada pendekatan berbasis aturan, dokumen direpresentasikan sebagai jumlah kemunculan setiap kata. Sedangkan pendekatan berbasis statistik mencoba mempelajari aturan klasifikasi sentimen dengan teknik-teknik pembelajaran mesin. Pada pendekatan berbasis statistik, teks direpresentasikan dalam bentuk numerik dan selanjutnya diproses menggunakan algoritma pembelajaran mesin. Penelitian pembelajaran mesin teranyar yang dilakukan pada bahasa Indonesia adalah penelitian oleh \parencite{CrisdayantiPurwarianti2019} untuk sentimen analisis dan oleh \parencite{Ibrohim_Budi_2019} untuk ujaran kebencian \& kasar.

Sebagian besar algoritma pembelajaran mesin memerlukan representasi teks dalam bentuk numerik. Untuk hal itu, berkembanglah berbagai teknik untuk merepresentasikan teks sebaik mungkin. Bentuk paling sederhananya adalah pendekatan \textit{one-hot-vector} yang merepresentasikan teks berdasarkan ada atau tidaknya saja. Representasi seperti ini memiliki kekurangan seperti tidak diperhatikannya letak kata dan membesarnya representasi kata seiring membesarnya kosa kata. Kekurangan ini dapat diselesaikan dengan \textit{word embedding} seperti Word2vec \parencite{MikolovWord2vec} yang mempelajari representasi kata sebagai vektor bernilai riil. Tetapi kelemahan pemrosesan teks dengan \textit{word embedding} adalah masih dangkalnya representasi. Representasi \textit{word embedding} tidak dapat menangkap interaksi antar kata di kalimat yang kompleks. Oleh karena itu, berkembanglah \textit{language model} seperti BERT \parencite{Devlin_Chang_Lee_Toutanova_2019} dan XLM \parencite{LampleConneau2019} yang tidak hanya belajar di level kata, tetapi sampai dapat memperhatikan konteks di mana kata tersebut berada. 

Pengembangan representasi teks yang akurat memerlukan banyak data teks. Nahasa Indonesia sebagai bahasa yang ingin diteliti di sini memiliki lebih sedikit data teks dibanding bahasa yang lebih populer seperti bahasa Inggris. Sebagai contoh, bahasa Inggris memiliki dataset seperti \textit{Amazon review}\footnote{\url{http://snap.stanford.edu/data/web-Amazon.html}} atau \textit{Yelp review}\footnote{\url{https://www.yelp.com/dataset/challenge}} yang totalnya hingga jutaan data. Banyaknya data ini mendorong perkembangan yang memungkinkan analisis sentimen mendapatkan akurasi 97.5\%\footnote{\url{https://gluebenchmark.com/leaderboard}} (Diakses pada tanggal 1 Mei 2020) seperti pada \textit{benchmark} GLUE\parencite{GLUE2019}.

Untuk menanggulangi masalah kurangnya data, dapat digunakan teknik yang disebut dengan transfer learning lintas bahasa. Transfer Learning adalah teknik melakukan pembelajaran mesin dari sebuah domain, biasanya yang memiliki lebih banyak data, lalu menggunakan model yang sudah dipelajari untuk menyelesaikan masalah di domain lainnya \parencite{ruder2019transfer}. Penggunaan teknik ini sangat sukses mendorong kemajuan besar di berbagai permasalahan pemrosesan teks alami. Dengan transfer learning lintas bahasa, bahasa yang memiliki sumber daya rendah dapat memanfaatkan sumber daya dari bahasa yang jauh lebih kaya.

Hasil penelitian \parencite{LampleConneau2019} membuktikan efektivitas \textit{transfer learning} lintas bahasa dengan \textit{language model} yang sudah dilatih pada Bahasa Inggris ke bahasa lainnya. Dalam penelitiannya, \textit{language model} dilatih pada berbagai bahasa secara \textit{unsupervised} atau tanpa korpus paralel sama sekali. Hasilnya terbukti efektif dalam berbagai masalah mulai dari translasi mesin, pengembangan language model bahasa yang memiliki data teks sedikit, hingga berbagai tugas klasifikasi. Penelitian ini mencoba menggunakan teknik tersebut untuk meningkatkan performa hasil penelitian \parencite{CrisdayantiPurwarianti2019} mengenai analisis sentimen di bahasa Indonesia dam versi biner penelitian \parencite{Ibrohim_Budi_2019} mengenai ujaran kebencian \& kasar. 

\section{Rumusan Masalah}

Berdasarkan latar belakang yang telah dipaparkan pada sub bab sebelumnya, tugas akhir ini akan fokus dalam mengetahui: 
\begin{enumerate}
	\item Bagaimana pengaruh teknik \textit{transfer learning} lintas bahasa dengan \textit{pretrained language model} dalam permasalahan analisis sentimen dan klasifikasi ujaran kebencian \& kasar bahasa Indonesia?
\end{enumerate}

\section{Tujuan}

Tujuan dari tugas akhir ini adalah sebagai berikut:
\begin{enumerate}
	\item Membangun model analisis sentimen dan klasifikasi ujaran kebencian \& kasar bahasa Indonesia menggunakan fitur dari \textit{language model pretraining} lintas bahasa.
	\item Membandingkan performa analisis sentimen dan klasifikasi ujaran kebencian \& kasar menggunakan \textit{language model pretraining} lintas bahasa dengan yang tanpa \textit{language model pretraining} lintas bahasa.
\end{enumerate}

\section{Batasan Masalah}

Batasan masalah diperlukan untuk membatasi dan menspesifikasi sejauh apa hasil tugas akhir ini akan dibuat. Berikut merupakan batasan masalah untuk tugas akhir ini:
\begin{enumerate}
	\item Bahasa yang akan dijadikan sumber pembelajaran adalah bahasa Inggris dan bahasa Indonesia.
	\item Data teks dan label untuk analisis sentimen menggunakan data dari penelitian sebelumnya pada topik ini oleh \parencite{CrisdayantiPurwarianti2019}.
	\item Data teks dan label untuk klasifikasi ujaran kebencian \& kasar menggunakan data dari penelitian sebelumnya pada topik ini oleh \parencite{Ibrohim_Budi_2019}.
\end{enumerate}

\section{Metodologi}

Berikut metodologi yang digunakan dalam pengembangan tugas akhir ini:
\begin{enumerate}
	\item Studi literatur \\
	Mempelajari berbagai literatur baik berupa buku, tesis, jurnal penelitian, makalah, maupun situs yang memuat informasi yang terkait dengan pembangunan analisis sentimen berbasis deep learning dan \textit{transfer learning} lintas bahasa. Studi literatur berfokus kepada \textit{transfer learning} lintas bahasa, analisis sentimen, deep learning, dan metrik evaluasinya.

	\item Analisis dan perancangan solusi \\
	Pada tahap ini, solusi terbaik akan dipilih dari alternatif solusi berdasarkan kelebihan dan kekurangan masing-masing solusi yang akan diimplementasikan pada analisis sentimen bahasa Indonesia dengan \textit{transfer learning} lintas bahasa. Perancangan solusi mencakup penyusunan arsitektur sistem dan perancangan komponen-komponen sistem yang akan dibangun dengan metode tertentu selama proses pengerjaan tugas akhir ini.

	\item Pengembangan dan evaluasi sistem \\
	Pada tahap ini, dilakukan eksperimen dari hasil analisis dan perancangan solusi, mengevaluasi/menguji  hasil eksperimen, dan mengoptimasi kinerja sistem berdasarkan hasil evaluasi. Keluaran utama dari tahap ini adalah analisis terhadap klasifikasi teks berbahasa Indonesia dengan model lintas bahasa.
\end{enumerate}

\section{Sistematika Pembahasan}
Berikut sistematika dan pembahasan tugas akhir ini:
\begin{enumerate}
	\item \textbf{Bab I Pendahuluan} berisi penjelasan yang melandasi pembuatan tugas akhir ini. Hal itu meliputi penjelasan latar belakang, rumusan masalah, tujuan, batasan masalah, metodologi, dan sistematika penulisan.
	\item \textbf{Bab II Studi Literatur} berisi hasil studi literatur terkait representasi teks dan \textit{language model} lintas bahasa. Hal ini meliputi penjelasan mengenai awal mula representasi teks lintas bahasa, perkembangan representasi lintas bahasa dengan \textit{shared sub-word vocabulary}, perkembangan \textit{language model}, dan penelitian terkaitnya.
	\item \textbf{Bab III Analisis dan Rancangan Analisis Sentimen Teks Berbahasa Indonesia Menggunakan Cross-Lingual Language Model Pretrain} berisi analisis permasalahan, analisis rancangan komponen dataset solusi, dan analisis rancangan komponen klasifikasi solusi.
	\item \textbf{Bab IV Eksperimen dan Pembangunan Sistem} berisi rincian proses mengenai eksperimen yang dilakukan pada tugas akhir beserta evaluasi dan analisis terhadap hasil eksperimen. Bab ini juga menjelaskan lebih rinci lingkungan dan implementasi masing-masing komponen beserta evaluasi terhadap hasil implementasi tersebut.
	\item \textbf{Bab V Kesimpulan dan Saran} berisi kesimpulan terhadap hasil penelitan yang telah dilakukan dalam tugas akhir beserta saran-saran terkait pekerjaan lanjutan yang dapat dijadikan sebagai acuan untuk pengembangan selanjutnya.
\end{enumerate}





