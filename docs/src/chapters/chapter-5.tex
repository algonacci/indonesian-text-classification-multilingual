\chapter{Kesimpulan dan Saran}

\section{Kesimpulan}
Berikut beberapa kesimpulan yang dapat ditarik dari tugas akhir ini:
\begin{enumerate}
    \item Model \textit{multilingual} XLM-R lebih bagus dibanding MultilingualBERT dalam membantu meningkatkan performa klasifikasi teks bahasa Indonesia. Representasi teks antar bahasa yang lebih bagus, baik melalui teknik optimisasi yang lebih optimal dan penambahan data yang XLM-R sudah lakukan, sangat berpengaruh terhadap kemampuan \textit{multilingual language model} berkinerja baik di permasalahan klasifikasi teks bahasa Indonesia.
    \item Penambahan dataset bahasa Inggris, terutama jika data bahasa Indonesia sedikit, dapat membantu meningkatkan performa klasifikasi teks bahasa Indonesia menggunakan \textit{multilingual language model} XLM-R.  Hanya ada kasus dimana penambahan data bahasa Inggris berlebih menurunkan performa klasifikasi. Analisis lebih lanjut mengenai kenapa hal ini terjadi dan apa solusinya diperlukan.
    \item Penggunaan \textit{multilingual language model} yang di \textit{fine-tune} sepenuhnya sangat efektif dalam klasifikasi teks bahasa Indonesia. 
    \begin{enumerate}
    	\item Pada eksperimen sentimen analisis dataset A, model mendapatkan F1-score 0,893. Sebuah peningkatan dari penelitian sebelumnya yang mendapatkan F1-score 0,8341.
        \item Pada eksperimen sentimen analisis dataset B, model mendapatkan F1-score sempurna. Sebuah peningkatan absolut dari penelitian sebelumnya yang mendapatkan F1-score 0,9369.
        \item Pada eksperimen klasifikasi ujaran kebencian, model mendapatkan F1-score 0.898 dan akurasi 89.9\%. Penelitian sebelumnya yang menggunakan 3 label, bukan yang disimplifikasi menjadi 2 seperti di penelitian ini, mendapatkan rata-rata akurasi tertinggi 77.36\%. Agar dapat dibandingkan, eksperimen dijalankan dengan konfigurasi label yang sama dan didapatkan rata-rata akurasi 89.52\% yang merupakan peningkatan dari penelitian sebelumnya.
    \end{enumerate}

\end{enumerate}

\section{Saran}
Berikut beberapa saran yang dapat digunakan untuk memperbaiki, memperbaharui, atau mengembangkan hasil tugas akhir ini:
\begin{enumerate}
    \item Jauhnya perbedaan, baik dari domain, bahasa, atau faktor lainnya yang mempengaruhi pengumpulan data, antara dataset dapat menyebabkan memburuknya performa \textit{multilingual learning}. Untuk penelitian selanjutnya, dapat dicoba beberapa cara untuk mengatasi hal ini. Beberapa diantaranya adalah seperti penelitian \parencite{Lai_Oguz_Yang_Stoyanov_2019} yang menggunakan \textit{universal data augmentation} untuk mengurangi perbedaan tadi pada saat pembelajaran dilakukan.
    \item Turunnya performa pada \textit{multilingual learning} permasalahan klasifikasi ujaran kebencian masih harus diteliti lebih lanjut. Perlu dianalisa lebih dalam lagi apakah hal ini dikarenakan teknik fine-tuningnya atau perbedaan domain yang melekat dalam dataset. Penelitian \parencite{Peters_Ruder_Smith_2019} meneliti perbedaan antara dua teknik fine-tuning yang dicoba pada tugas akhir ini dan mencoba menganalisa kemiripan datasetnya. Hal tersebut dapat dijadikan pedoman dalam menganalisa lebih lanjut fenomena yang diobservasi pada tugas akhir ini.
    \item Tugas akhir ini sudah membuktikkan efektifnya penggunaan \textit{language model} dalam berbagai permasalahan klasifikasi teks bahasa Indonesia. Penelitian \parencite{Conneau_XLMR} telah mengobservasi turunnya performa model secara general dengan ditambahnya bahasa dalam pelatihan \textit{multilingual language model} Untuk sampai saat tugas akhir ini ditulis, belum terdapat \textit{language model} spesifik yang dilatih secara masif dalam bahasa Indonesia. Hal ini dapat dicoba dan dibandingkan performanya dengan \textit{multilingual language model} yang dilatih dalam berbagai bahasa. 
\end{enumerate}