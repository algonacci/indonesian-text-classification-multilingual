\chapter{Kesimpulan dan Saran}

\section{Kesimpulan}
Berikut beberapa kesimpulan yang dapat ditarik dari tugas akhir ini:
\begin{enumerate}
    \item Penambahan dataset bahasa Inggris, terutama jika data bahasa Indonesia sedikit, dapat membantu meningkatkan performa klasifikasi teks bahasa Indonesia menggunakan \textit{multilingual language model}. Hanya saja, disparitas antara data bahasa Indonesia dengan data bahasa Inggris, yang dapat dilihat dari performa \textit{zero-shot} dan \textit{monolingual baseline}-nya, harus diperhatikan. Pada kasus analisis sentimen dengan dataset Prosa dan Trip Advisor, jarak tersebut masing-masing rata-rata 0.0005 dan 0.044. Sehingga eksperimen sentimen analisis skenario 3 pada data tersebut mendapat peningkatan F1-score masing-masing rata-rata 0.229 dan 0.107. Sedangkan pada klasifikasi ujaran kebencian yang jaraknya 0.093, penambahan data bahasa Inggris menjadi kurang bermanfaat dan bahkan menurunkan performa model.
    \item Penggunaan \textit{multilingual language model} yang di \textit{fine-tune} sepenuhnya sangat mangkus dalam klasifikasi teks bahasa Indonesia. Pada eksperimen sentimen analisis dataset A, model mendapatkan F1-score 0.886. Sebuah peningkatan dari penelitian sebelumnya yang mendapatkan F1-score 0,8341. Pada eksperimen sentimen analisis dataset B, model mendapatkan F1-score sempurna. Sebuah peningkatan absolut dari penelitian sebelumnya yang mendapatkan F1-score 0,9369. Pada eksperimen klasifikasi ujaran kebencian, model mendapatkan F1-score 0,892 dan akurasi 89.4\%. Penelitian sebelumnya yang menggunakan 3 label, bukan yang disimplifikasi menjadi 2 seperti di penelitian ini, mendapatkan \textit{average accuracy} tertinggi 77.36\%. Agar dapat dibandingkan, eksperimen dijalankan dengan konfigurasi yang sama dan didapatkan \textit{average accuracy} 89.52\%. 

\end{enumerate}

\section{Saran}
Berikut beberapa saran yang dapat ditarik dari tugas akhir ini:
\begin{enumerate}
    \item Terdapatnya disparitas antara dataset dapat menyebabkan turunnya performa. Untuk penelitian selanjutnya, dapat dicoba beberapa cara untuk mengatasi hal ini. Beberapa diantaranya adalah seperti penelitian \parencite{Lai_Oguz_Yang_Stoyanov_2019} yang menggunakan \textit{universal data augmentation} untuk mengurangi jarak tadi.
\end{enumerate}